\subsection{Opportunities and challenges of use of machines}
The use of computers in mathematics is widespread and likely to increase.\\
\hfill\\
This presents several \textbf{opportunities}:
\begin{itemize}
\item Personal assistant: Guiding intuition, checking hypotheses
\item Theorem proving: Large computations, checking many cases
\item Theorem checking: Formal verification
\end{itemize}
But equally it presents several \textbf{challenges}:
\begin{itemize}
\item Are computations rigorous?
\item Are proofs with computation understandable to humans?
\item How do we find errors?
\end{itemize}


\subsection{My use of computation}
I use computation daily to guide my proofs and intuition.
Two particular kinds of computations occur in my proofs.
\par
The first is when I want to understand the spectrum of infinite dimensional operator, but of course computers can't handle these as such. Then I consider well-chosen finite-dimensional subspace and do explicit computations there.
\begin{itemize}
\item Do non-rigorous computation to guess good answers, then compute answers using exact arithmetic.
\item Happy trade-off between quality of numerical result and computation time.
\end{itemize}%\pause
The second is when, after doing theoretical manipulations, I need to show that some messy explicit integral is less than 1.
\begin{itemize}
\item My computations are non-rigorous!
\item Very difficult to referee --- many potential sources for error.
\end{itemize}
