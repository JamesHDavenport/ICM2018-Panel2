\subsection{A (very brief, partial) history}
\begin{description}
	\item[1963]``Solvability of Groups of Odd Order'': 254 pages\footnote{``one of the longest proof to have appeared in the mathematical literature to that point.'' \cite{Gonthieretal2013a}.} \cite{FeitThompson1963}. Also Birch \& Swinnerton--Dyer published \cite{BirchSwinnertonDyer1963}, the algorithms underpinning their conjectures.
\item[1976]``Every Planar Map is Four-Colorable'': 256 pages + computation \cite{AppelHaken1976a}.
\item[1989]Revised Four-Color Theorem proof published \cite{AppelHaken1989}.
\item[1998]Hales announced proof of Kepler Conjecture.
\item[2005]Hales' proof published in an abridged form ``uncertified''\footnote{\emph{Mathematical Reviews} states ``Nobody has managed to check all the details of the proof so far, but the theoretical part seems to be correct. The whole proof is considered and assumed to be correct by most of the mathematical community.'' \url{https://mathscinet.ams.org/mathscinet-getitem?mr=2179728}.} \cite{Hales2005}.
\item[2008]Gonthier stated formal proof of Four-Color Theorem \cite{Gonthier2008}.
\item[2012]Gonthier/Th\'ery stated\footnote{``Both the size of this proof and the range of mathematics involved make formalization
a formidable task'' \cite{Gonthieretal2013a}.} formal proof of Odd Order Theorem \cite{GonthierThery2012a,Gonthieretal2013a}.
\item[2013]Helfgott published (arXiv) proof of ternary Goldbach Conjecture \cite{Helfgott2013a}.
\item[2014]Flyspeck project announced formal proof of Kepler Conjecture \cite{Hales2014a}.
\item[2015]Maynard published ``Small gaps between primes'' \cite{Maynard2015a}.
\item[2017]Flyspeck paper published \cite{Halesetal2017a}.
\end{description}
The Odd Order Theorem is important, but chiefy because it leads to the classification of finite simple groups. One might ask when this will be formally proved, and indeed I did ask Georges Gonthier this question. He answered that he worked, not so much from \cite{FeitThompson1963} itself as from \cite{Benderetal1994,Peterfalvi2000}, two substantial books which between them described much work simplifying  and clarifying the argument, and that such work had yet to be done for the full classification.
\subsection{Questions for Consideration}
What are the implications for
\begin{itemize}
\item authors
\item journals and their publishers
\item the refereeing process (we note that, although \cite{Hales2005} took seven years not to be completely refereed,  \cite{Halesetal2017a} still took three years to be refereed: ``formal'' is not the same as ``simple''.) % hales at IMS 2018: see JHD's notes.
\item readers
\item the storage and curation of such proofs, and, if necessary, the software necessary to run such proofs.?% new since panel
\end{itemize}
These questions are not independent: the refereeing process is run by journals, and one can ask whether the journal should keep the machine-readable proof, as with \cite{Hales2005}, or whether Helfgott is right with ``available on request'', or maybe Maynard's ``at \url{www.arxiv.org}''.

