\subsection{A (very brief, partial) history}
\begin{description}
\item[1963]``Solvability of Groups of Odd Order'': 254 pages\footnote{``one of the longest proof to have appeared in the mathematical literature to that point.'' \cite{Gonthieretal2013a}.} \cite{FeitThompson1963}.
\item[1976]``Every Planar Map is Four-Colorable'': 256 pages + computation \cite{AppelHaken1976a}.
\item[1989]Revised Four-Color Theorem proof published \cite{AppelHaken1989}.
\item[1998]Hales announced proof of Kepler Conjecture.
\item[2005]Hales' proof published ``uncertified'' \cite{Hales2005}.
\item[2008]Gonthier stated formal proof of Four-Color Theorem \cite{Gonthier2008}.
\item[2012]Gonthier/Th\'ery stated\footnote{``Both the size of this proof and the range of mathematics involved make formalization
a formidable task'' \cite{Gonthieretal2013a}.} formal proof of Odd Order Theorem \cite{GonthierThery2012a,Gonthieretal2013a}.
\item[2013]Helfgott published (arXiv) proof of ternary Goldbach Conjecture \cite{Helfgott2013a}.
\item[2014]Flyspeck project announced formal proof of Kepler Conjecture \cite{Hales2014a}.
\item[2016]Maynard published ``Large gaps between primes'' \cite{Maynard2016a}.
\item[2017]Flyspeck paper published \cite{Halesetal2017a}.
\end{description}
The Odd Order Theorem is important, but chiefy because it leads to the classification of finite simple groups. One might ask when this will be formally proved, and indeed I did ask Georges Gonthier this question. He answered that he worked, not so much from \cite{FeitThompson1963} itself as from \cite{Benderetal1994,Peterfalvi2000}, two substantial books which between them described much work simplifying  and clarifying the argument, and that such work had yet to be done for the full classification.
\subsection{Questions for Consideration}
What are the implications for
\begin{itemize}
\item authors
\item journals and their publishers
\item the refereeing process
\item readers
\item the storage and curation of such proofs?% new since panel
\end{itemize}

