The awareness of the mathematical community towards the use of computers in
proofs of mathematical results started in 1976 with the announcement of the
proof of the Four-Color Theorem~\cite{AppelHaken1976a}.

However, there are earlier examples of proofs that were partially or completely
done by a machine. An interesting example is Floyd and Knuth's proof of
optimality of the 16-comparator sorting network on 7 inputs (Theorem 5
in~\cite{FloydKnuth1973}), which starts:
\begin{quote}
This theorem was proved by exhaustive enumeration on a CDC G-21 computer at
Carnegie Institute of Technology in 1966.
\end{quote}

Optimality of sorting networks is an interesting problem in combinatorics. The
question we want to answer is: what is the length $S(n)$ of the shortest
sequence of compare-and-swap operations (i.e., atomic operations consisting of
sorting a pair of values) that will sort all inputs of a given length $n$?
Floyd and Knuth's work cited above addresses this problem for $2\leq n\leq 7$.
Except for the values of $S(4)$ and $S(6)$, which are derived from those of
$S(3)$ and $S(5)$ by application of a more general theorem, all values are
derived by exhaustively enumerating the possible sequences of length $S(n)-1$
and showing that, for each of them, there is an input of length $n$ that they do
not sort. As the authors note in their conclusion, this method is however
``quite unsatisfactory for higher values of $n$'': the number of cases that need
to be analyzed for establishing $S(7)$ is unmanageable for a human being, and
lay at the limits of what could be computed in 1966.

\paragraph{Two styles of proving.}

Floyd and Knuth's proof, as well as the proof of the Four-Color Theorem, are
examples of one family of machine proofs: they establish a property by means of
an \emph{ad-hoc} computer program that is written for that specific purpose.
Nowadays, this approach is not very common: verifying such proofs also demands
that the program be verified, a task that does not fall under the usual scope of
the peer-reviewing process. Instead, it is generally considered preferable to
use a \emph{theorem prover}: a general-purpose program that can construct and/or verify proofs in a particular logic.

Trusting a proof produced with the help of a theorem prover also has some
implications. First, one still needs to trust a computer program -- the theorem
prover itself. The difference with respect to using \emph{ad-hoc} programs is
that we are now considering a program that has been subject to much wider
scrutiny, and it can be reasonably argued that the ensuing proof is as
trustworthy as a published mathematical proof that has been subject to
peer-reviewing. The other point that needs to be checked is that the encoding of
the actual mathematical problem in the theorem prover's logic is correct: this
is generally accepted as part of the peer-reviewing process, as these encodings
are typically included and discussed in submitted articles. (This aspect is also
an issue when using \emph{ad-hoc} computer programs in proofs, but in that
scenario the encodings used tend to be much more direct.)

Theorem provers come in a wide variety of styles and flavors, ranging from very
general purpose to more specifically tailored for a particular family of
problems. They use different logics, and are often uncomparable in their
usefulness. A very encompassing overview of the world of theorem provers can be
found in~\cite{Wiedijk2006}.

\paragraph{Machine-assisted proofs today.}

Forty years after the announced proof of the Four-Color Theorem,
machine-assisted proofs are everywhere. Arguably, their widest area of
application is hardware and software verification, of which Floyd and Knuth's
problem is an example. In an area where we are increasingly dependent on
computers to perform critical tasks -- from controlling air traffic to
administering medicine to patients -- it is more and more important that there
be no errors in the execution of those tasks, whether due to programming errors
or to hardware flaws.

As regards hardware, many useful properties can be verified in a fully automated
way~\cite{Kropf1999}. As such, formal verification is becoming more commonplace
in industry. Software verification tends to be more complex, but the number of
success stories is increasing. In recent years there have been large projects
addressing formalization of large fragments of widely-used programming
languages, making it possible for non-experts to experiment with proving
properties of the programs they write.

Much more challenging is the formalization of mathematical proofs in a
computer. These proofs tend to be much less mechanic and systematic, requiring
constant interaction between the computer and an expert -- where the expert
``guides'' the computer through the main steps of the proof. Mathematics also
tends to build upon itself, and researchers often find that the biggest
challenge in formally verifying an ``interesting'' result lies on the
unavailability of libraries formalizing the relevant underlying theories.
Nevertheless, recent formalizations of e.g.~the proof of the Four-Color
Theorem~\cite{Gonthier2008} and the proof of Kepler's
conjecture~\cite{Hales2014a} show that this goal is not hopeless. (A recent
contribution to this list is a formal proof of Floyd and Knuth's results on
sorting networks~\cite{lcfEtAl2017a}, along with the recently established value
of $S(9)$~\cite{CodishEtAl2016}.)

A different approach to machine-verified proofs of mathematical results, which
some may argue is less elegant, capitalizes on the sucess of SAT solvers. A SAT
solver is a program that tackles the Boolean satisfiability problem: given a
propositional formula, decide whether it has a satisfying assignment. Although
this problem is NP-complete, years of intensive investiment in researching
efficient techniques for solving it has resulted in extremely powerful tools
that can solve formulas with millions of variables and billions of clauses. To
test the limits of these programs, several researchers have experimented with
encoding open mathematical problems (typically from combinatorics) as
propositional formulas, which were then successfully proven to be
unsatisfiable~\cite{HeuleKullmann2017}.

Until recently, SAT solvers were viewed by some with skepticism: their
complexity made them impossible to analyze in practice, and in the case a
formula was claimed to be unsatisfiable no independently verifiable guarantee of
this fact was provided. The situation changed recently, and the majority of
today's SAT solvers also output traces that allow independent, formally verified
systems to check proofs of unsatisfiability of propositional
formulas~\cite{lcfEtAl2017b,lcfEtAl2017c}.
