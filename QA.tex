\begin{description}
\item[Q]Isn't there a problem with proprietary systems (e.g. Magma, Maple or Mathematics). You might not be able to find referees who have them.
\item[A]There is a problem here, but in fact you are more likely to find a referee who knows (and has) one of these than you are some rare open-source package. The real problem is the proprietary nature of the algorithms. ``Yes, this large piece of code, which may be proprietary or open-source, gives me this result, but do I trust it?''  For computer algebra, the question is discussed in \cite{Davenport2017z}.
\item[Q]What about a system that produces verified code, which is then compiled and run?
\item[A]That's a good question, and there is some of this in Flyspeck \cite{Halesetal2017a}.  See also Paulson's MetiTarski project \cite{AkbarpourPaulson2010}.
Of course, one would need a verified compiler, but such things exist these days.
\item[Q]Is there much consistency between journals on how these proofs are treated?
\item[A]There isn't even much consistency within a given journal. \cite{Hales2005} and \cite{Maynard2015a} were both in \emph{Annals of Mathematics}, yet seem to have been treated differently. There is no caution on \cite{Maynard2015a}, and it was published much more rapidly than  \cite{Hales2005}. Conversely the computer programs underpinning \cite{Hales2005} are on the \emph{Annals of Mathematics} website, whereas \cite{Maynard2015a} simply says ``An ancillary Mathematica(R) file detailing these computations is available alongside this
paper at \url{ www.arxiv.org}''.
\end{description}
